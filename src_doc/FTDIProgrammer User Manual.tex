%	bullet and ordered list 
% \begin{itemize}
% \item \blindtext
% \item \blindtext
% \end{itemize}
% \begin{enumerate}
% \item \blindtext
% \item \blindtext
% \end{enumerate}
% \begin{description}
% \item [Ant] \blindtext
% \item [Elephant] \blindtext
% \end{description}


\documentclass[a4paper,12pt]{report}% azczxc

% \documentclass{article}% use option titlepage to get the title on a page of its own.
% \usepackage{blindtext}
% \usepackage{hyperref}
\title{FTDI Programmer User Manual}
\date{\today}
\author{Kunal Chakraborty\\ kunalc@cdot.in}

\setlength{\parindent}{0pt}
\setlength{\parskip}{1em}

\begin{document}
\maketitle

%% table of content ... list of fig...


\section{Introduction}\label{intro}	% to be used with \ref{intro}

FTDI-programmer is an application written in pure python to access/program on board devices through FTDI devices 
using JTAG/SPI/I2C/GPIO interfaces. JTAG programming is done through svf file. It can be used in place of any 
external emulator. The application is built on PyFtdi driver. %--- url(url = https://github.com/eblot/pyftdi) 



The source code is compatible with both Windows and Linux system. However this manual is written from a windows 
user's point of view.


This is a Beta %\beta
version and hence the application has couple of limitations as listed in section \ref{sec:limit} 

	

\section{Features}
\begin{itemize}
	\item
	On board device programming through JTAG interface. It needs svf file for JTAG programming.
	
	\item
	TI UCD device programming through I2C interface. It needs SMBUS csv file programming. This file can be
	generated by Fusion tool.
	
	\item
	SPI Flash programming.(to be implemented)
	
	\item
	On board device access thhrough SPI/I2C/GPIO interfaces. (to be implemented)

\end{itemize}	

	

\section{Limitation}\label{sec:limit}
\begin{enumerate}
	\item I2C mode
		\begin{enumerate}
			\item 
			FTDI devices does not support multimaster and clock stretching. Hence this application will not work with
			slave devices which requires clock stretching. In that case application can be run on lower frequency
			so that target device may get enough time to respond. However there is a workaround given by FTDI by 
			connecting I2C SCL to a separate gpio of FTDI device so that FTDI device can read the clock line. This 
			feature is not supported in present version FTDI-programmer.

			\item	
			Highest I2C slave address supported is 0x78. It is the limitation from pyftdi driver. No workaround is provided.
		\end{enumerate}
	
	\item	JTAG mode
		\begin{enumerate}
			\item Target device must be the only device in JTAG chain because SVF parser of FTDI-programmer does not support
			header and trailer addition. This is kept as Future Development.
			\item
			SVF file verification takes longer than expected. This is because pyftdi driver takes long time to send data to 
			application after reading from usb. This is kept as Future Development.
			\item
			Only Max-V CPLD programming is supported. If any other device is detected, FTDI-programmer will flag a warning
			and continue with programming. However the integrity of the programming is not validated.  
		\end{enumerate}

	\item	Others
		\begin{enumerate}
			\item 
			MS Windows detects every channel of FTDI device as a separate usb device with same VID and PID. Hence FTDI-programmer
			can access only single channel of FTDI device. So user must enable a single lib-usb driver as per FTDI channel number
			as discussed in section \ref{sec:install}
		\end{enumerate}
	\end{enumerate}	


\section{Installation}\label{sec:install}
	All the required python packages are compiled and supplied with FTDI-programmer. However it needs a low level backend driver
	libusb to talk to FTDI device. You must have admin privilege to carry out below steps.
	Step by step installation guide is given below.
	\begin{description}
		\item [1. libusb installation]
		An easy way to install libusb backend on windows is Zadig. Download the latest version of Zadig %url = https://zadig.akeo.ie/
		Run zadig.exe and you should get a dialog as shown below
			% figure -----
		Go to Options -- List All Devices. Then select FTDI-channel0 from drop down option. Next select libusb-win32 as driver.
			% figure -----
		Finally, click install driver.
		Follow the same step to install driver on FTDI-channel1
		
		\item [2. Get FTDI-programmer]
		Download a fresh copy of FTDI-programmer from 'https://github.com/kunalcdot/FTDI-programmer'. The application is 
		under /bin folder. Run FTDI-programmer.exe to start the application.
		
	\end{description}
	



\section{Programming Guide}\label{sec:prog_guide}
	\subsection{JTAG Programming}
	pre requisite -- svf file
	zadig channel selection
	% figure
	
	
	\subsection{UCD Programming}
		\subsubsection{SMBUS Flash script generation}
			connect ti fusion tool.. 
			% figure.. fusion
			% figure.. sample csv file
			
		\subsubsection{USB Channel Selection}
			zadig channel selection
			% figure
		
		\subsubsection{Run FTDI-programmer}
		
			% fig...
		
		
\section{Future Development}
\begin{enumerate}
	\item 
	All programming file parser to be updated with 'regular expression' module. It should support all svf standard command.
	
	\item 
	JTAG read time needs to be reduced.
	
	\item 
	SPI Flash Programming option to be developed
	
	\item 
	An utility needs to be provided to read/write on board devices through I2C/SPI/GPIO etc

\end{enumerate}

\section{Troubleshooting}
Ensure only one FTDI device is connected and single ftdi channel is enabled. Refer to section ~\ref{install} for more details.
% abc \ref{install}


\end{document}